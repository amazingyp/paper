% !TeX root = ../main.tex

\ustcsetup{
  keywords  = {对象存储,纠删码,文件分配模式},
  keywords* = {object storage, erasure coding, file allocation mode},
}

\begin{abstract}
在新数据时代,电子商务和社交网络的兴盛导致数据在网络上传播更加迅速,以视频和图片为首的非结构化数据的数量飞速增长,传统的单点存储方式由于无法实现高可用、易扩展等特性,已不能应对各行各业的应用对数据存储的需求,对象存储应运而生。对象存储因其低成本、高扩展和大容量的特点成为存储领域的重要的存储方式,受到越来越多的用户的重视。当前的对象存储系统虽然能很好地解决非结构化数据的存储问题,但现有系统存在一些影响性能的问题。

首先,系统的元数据映射简单,随着文件数量增长,元数据存储存在压力,读写性能也会下降;其次,系统的冗余存储存在瓶颈,如果系统直接使用纠删码存储则需要写入时计算,性能较差,如果先用多副本再转纠删码存储,则需要对文件进行迁移,文件被读写两次,系统的开销大;最后,现有系统底层读写都利用Linux的文件系统,当文件增多时,维护和读取索引文件的开销会增大。

为了应对海量的非结构化数据,本系统以低成本和大量小对象存储为目标,提出了基于条带的分配模式,通过将对象数据以条带的方式分配,增加了对象到条带的映射,同时利用条带内各块之间的内在关系,极大地减少了元数据的数据量。为了平衡系统的性能与成本,本系统使用了多副本和纠删码结合的冗余方式,条带的设计兼顾了文件多副本形式和纠删码形式的存储,能够减少多副本存储转纠删码存储时的额外的数据读写。本文还设计了一种和条带分配模式相适配的磁盘划分方案,应用此方案可以将磁盘直接挂载在操作系统中,通过预设的接口读取对象数据,减少了通过操作系统读取对象数据的开销。

本文详细介绍了基于条带的分配模式,在这样的分配模式下实现了一个分布式对象存储系统,目前该系统已经成为部门的核心应用。这篇文章可以为解决对象存储现有问题提供思路,也可以为其他对象存储系统的设计与实现提供参考性建议。


\end{abstract}

\begin{abstract*}
In the new data era, the prosperity of e-commerce and social networks has led to the rapid spread of data on the network. The number of unstructured data, led by videos and pictures, has grown rapidly. Because the traditional storage method using a single node to save data can not achieve high availability, easy expansion and other characteristics. So it is hard to meet the data storage needs of applications in all walks of life. At that time, object storage emerged. Object storage has become an important storage method in the storage field due to its low cost, high expansion and large capacity, and has attracted more and more attention. Although the current object storage system can well solve the storage problem of unstructured data, there are some problems affecting the performance of the system.

Firstly, the metadata mapping of the system is simple, and as the number of files increases, there is pressure on metadata storage, resulting in a decrease in read and write performance. Secondly, there is a bottleneck in the redundant storage of the system. If the system directly uses erasure coding as the redundant scheme, it requires calculation when writing, which results in poor write performance. If a file is replicated first and then converted to be erasure coding, the file needs to be migrated, and is read and written twice, resulting in high system overhead. Finally, the existing system utilizes the Linux file system for low-level reads and writes, and as the number of files increases, the cost of maintaining and reading index files increases.

In order to cope with massive amounts of unstructured data, this system aims to store large amounts of small files with low cost, and thus proposes a file allocation mode based on stripes. By allocating files to stripes, it adds the mapping of files to stripes, and utilizes the internal relationships among the blocks within the same stripes, greatly reducing the amount of metadata. In order to balance the performance and cost of the system, this system uses a redundant approach combining replication and erasure coding. The design of stripes balances takes account of replication and erasure coding at the same time, which can reduces the additional read and write required for converting files replication to erasure coding. This thesis also designs a disk partitioning scheme that is compatible with the stripe allocation mode. By applying this scheme, the disk can be directly mounted on the operating system, and object data can be read through a preset interface, reducing the overhead of reading files through the operating system.

This thesis introduces file allocation mode based on strips in detail, and implements a corresponding distributed object storage system. At present, this system has become the core application of the department. This thesis can provide ideas for solving the existing problems of object storage, and also provide reference suggestions for the design and implementation of other object storage systems.


\end{abstract*}
